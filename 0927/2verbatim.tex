\documentclass{article}
\usepackage[magyar]{babel}
\usepackage{t1enc}
\usepackage{lipsum}
\usepackage{float}

\newfloat{forraskod}{hbt}{lop}[section]

\begin{document}
    egy sor \verb | random karakterek */~ˇˇ() \textbf{hello} | folytatás \verb | második verbatim :,? | sima szöveg \verb | még egy %!+- | újra rendes
    
    \listof{forraskod}{Tartalomjegyzék}
    \clearpage

    \lipsum[1]
    
    \begin{forraskod}
        \begin{verbatim}
            Tétel környezet kódja:
                Preambulum:
                    \newtheorem{típus}[számozás módja]{név}
                Document:
                    \begin{típus}
                        Szöveg...
                    \end{típus}
        \end{verbatim}
        \caption{Felirat 1.}
    \end{forraskod}
    
    \lipsum[2-3]
    
    \begin{forraskod}
        \begin{verbatim}
            Lista kódja:
                \begin{enumerate}
                    \item egy
                    
                    \begin{itemize}
                        \item második szint
                    \end{itemize}
                    
                    \item kettő
                \end{enumerate}
        \end{verbatim}
        \caption{Felirat 2.}
    \end{forraskod}
    
\end{document}