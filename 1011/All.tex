\documentclass{article}
\usepackage[magyar]{babel}
\usepackage{t1enc}
\usepackage{enumitem}
\usepackage{amsmath}
\usepackage{amssymb}
\usepackage{mathtools}
\usepackage{xcolor}

\begin{document}

    \begin{enumerate}[label=\alph*)]
    \item 
        Az $ \frac{1}{n^2} $ sorösszege:
            \[ \sum_{n=1}^{\infty} \frac{1}{n^2} = \frac{\pi^2}{6}. \]
    
    \item
        Az $n!\text{ }(n \text{ faktoriális})$ a számok szorzata 1-től $n$-ig, azaz
            \begin{equation}
                n! := \prod_{k=1}^n k = 1\cdot2\cdot...\cdot n.
            \end{equation}
        Konvenció szerint 0! = 1.
    
    \item
        Legyen $0 \leq k \leq n. $ A binomiális együttható
            \[ \binom{n}{k} := \frac{n!}{k! \cdot (n-k)!}, \]
        ahol a faktoriálist (\textcolor{red}{1}) szerint definiáljuk.
    
    \item
        Az előjel- azaz szignum függvényt a következőképpen definiáljuk:
            \[ sgn(x) := \begin{cases}
                            1, & \text{ha } x > 0, \\
                            0, & \text{ha } x = 0, \\
                            -1, & \text{ha } x < 0.
                         \end{cases}\]
                     
    \end{enumerate}
    
    \begin{enumerate}[label=\alph*)]
    \item 
        Legyen
            \[ [n] := \{ 1, 2, \cdot \cdot \cdot , n \}\]
        a természetes számok halmaza 1-től $n$-ig.
    
    \item
       Egy $n$-edrendű \textit{permutáció} $\sigma$ egy bijekció $[n]$-ből $[n]$-be. Az $n$-edrendű permutációk halmazát, az ún. szimmetrikus csoportot, $S_n$-nel jelöljük.
    
    \item
        Egy $\sigma \in S_n$ permutációban inverziónak nevezünk egy $(i, j)$ párt, ha $i < j$
        de $\sigma_i > \sigma_j.$
    
    \item
       Egy $\sigma \in S_n$ permutáció paritásának az inverziók számát nevezzük:
        \[ \mathcal{I}(\sigma) := \Bigl\vert \{(i, j) | i, j \in [n], i < j, \sigma_i > \sigma_j\} \Bigr\vert . \]

    \item
        Legyen $ A \in \mathbb{R}^{n\times n} $, egy $n \times n$-es (négyzetes) valós mátrix:
                     \[ A = \left( \begin{matrix}
                                    a_11 & a_12 & \cdots & a_1n \\
                                    a_21 & a_22 & \cdots & a_2n \\
                                    \rotatebox{90}{$\cdots$} & \rotatebox{90}{$\cdots$} & \rotatebox{135}{$\cdots$} & \rotatebox{90}{$\cdots$}\\
                                    a_n1 & a_n2 & \cdots & a_nn \\
                                \end{matrix} \right) \]
        Az $A$ mátrix determinánsát a következőképpen definiáljuk:
        \begin{equation}
            A = \left\vert \begin{matrix}
                                    a_11 & a_12 & \cdots & a_1n \\
                                    a_21 & a_22 & \cdots & a_2n \\
                                    \rotatebox{90}{$\cdots$} & \rotatebox{90}{$\cdots$} & \rotatebox{135}{$\cdots$} & \rotatebox{90}{$\cdots$}\\
                                    a_n1 & a_n2 & \cdots & a_nn \\
                                \end{matrix} \right\vert 
                                := \sum_{a \in S_n} (-1)^{\mathcal{I}(\sigma)} \prod_{i=1}^n a_{i\sigma_i}
        \end{equation}
    \end{enumerate}
    
    Tekintsük az $L = \{0, 1\}$ halmazt, és rajta a következő, igazságtáblával definiált műveleteket:
    \[
    \begin{array}{c||c}
        x & \overline{x} \\ \hline
        0 & 1 \\
        1 & 0
    \end{array}
    \quad
    \begin{array}{cc||c|c|c}
        x & y & x \vee y  & x \wedge y  & x \rightarrow y \\ \hline
        0 & 0 & 0 & 0 & 1 \\
        0 & 1 & 1 & 0 & 1 \\
        1 & 0 & 1 & 0 & 0 \\
        1 & 1 & 1 & 1 & 1 \\
    \end{array}
    \]
    Legyenek $a, b, c, d \in L.$ Belátjuk a következő azonosságot:
    \begin{equation}
        (a \wedge b \wedge c) \rightarrow d = a \rightarrow \bigl(b \rightarrow (c \rightarrow d)\bigr)
    \end{equation}
    
    A következő azonosságokat bizonyítás nélkül használjuk:
    \begin{subequations}
        \begin{equation}
            x \rightarrow y = \overline{x} \vee y
        \end{equation}
        \begin{equation}
            \overline{x \vee y} = \overline{x} \wedge \overline{y} 
            \quad
            \overline{x \wedge y} = \overline{x} \vee \overline{y}
        \end{equation}
    \end{subequations}
    
    A (3) bal oldala, (4) felhasználásával
    \begin{equation}
        (a \wedge b \wedge c) \rightarrow d \underset{(4a)}{=} \overline{a \wedge b \wedge c} \vee d \underset{(4b)}{=} (\overline{a} \vee \overline{b} \vee \overline{c}) \vee d.
    \end{equation}
        
    A (3) jobb oldala, (4a) ismételt felhasználásával
    \begin{align}
        \nonumber a \rightarrow (b \rightarrow (c \rightarrow d)) &= \overline{a} \vee \bigl(b \rightarrow (c \rightarrow d)\bigr) \\
        &= \overline{a} \vee \bigl(\overline{b} \vee (c \rightarrow d)\bigr) \\
        \nonumber &= \overline{a} \vee \bigl(\overline{b} \vee (\overline{c} \vee d)\bigr),
    \end{align}
    ami a $\vee$ asszociativitása miatt egyenlő (5) egyenlettel.
    
    \begin{subequations}
        \begin{align}
        (a+b)^{n+1} &= (a+b) \cdot \left( \sum_{k=0}^n \binom{n}{k} a^{n-k}b^k \right) \\
        \nonumber &= \cdots \\
        &= \sum_{k=0}^n \binom{n}{k} a^{(n+1)-k}b^k + \sum_{k=1}^{n+1} \binom{n}{k-1} a^{(n+1)-k}b^{k} \\
        \nonumber &= \cdots \\
        \begin{split}
            &=\binom{n+1}{0} a^{n+1-0} b^0 + \sum_{k=1}^n \binom{n+1}{k} a^{(n+1)-k}b^k \\
            &+ \binom{n+1}{n+1} a^{n+1-(n+1)} b^{n+1}
        \end{split} \\
        &= \sum_{k=0}^{n+1} \binom{n+1}{k} a^{(n+1)-k}b^k
    \end{align}
    \end{subequations}
    
\end{document}