\documentclass{book}
\usepackage[magyar]{babel}
\usepackage{t1enc}
\usepackage{lipsum}
\usepackage{xcolor}
\usepackage[backref=section, citebordercolor=white, citecolor=green, linkbordercolor=white, linkcolor=blue]{hyperref}
\usepackage{imakeidx}
\usepackage{tocbibind}

\makeindex
\makeindex[name=people, title=Második]
\renewcommand{\bibname}{Források}

\begin{document}

    \frontmatter
        \title{Könyv}
        \author{Gyáni Kevin Zsolt}
        \maketitle
    
        \tableofcontents
        \index{tartalomjegyzek@\textit{tartalomjegyzék}}
    
    \mainmatter
        \chapter{Első fejezet}
            \index[people]{fejezet vége|)}
            \cite{abrharber03}
            \index{elso@első|see{második, albejegyzés2}}
            \lipsum[2-7]
            \index{dupla}
            \cite{knuth84}
        \chapter{Második fejezet}
            \lipsum[3-5]
            \index{második}
            \index{második@albejegyzés1}
            \index{második@albejegyzés2}
            \index{második@albejegyzés2@alalbejegyzés}
        \chapter{Harmadik fejezet}
            \lipsum[1]
            \index{dupla}
            \cite{g4gbin}
            \index[people]{random}
            \lipsum[8]
            \index{harmadik|textbf}
            
    \backmatter
        \printindex
    
        \bibliographystyle{plain}
        \bibliography{ltx.bib}
        
        \printindex[people]
    
\end{document}